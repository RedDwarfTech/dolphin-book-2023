\documentclass[../../../dolphin-book-2023.tex]{subfiles}

\begin{document}

\subsection{Rust错误处理}

\subsubsection{主流的错误处理方式}

Rust在设计时,肯定参考了其他语言错误处理的优点和缺点。在了解Rust的错误处理方式之前,有必要了解目前所有语言的错误处理主流类型。看看每一种错误处理的方式有哪些优点和劣势。从而了解为什么要这么设计,在选择时能做出更加合适的决策。

\paragraph{使用返回值(错误码)}

使用返回值来表征错误,是最古老也是最实用的一种方式,它的使用范围很广,从函数返回值,到操作系统的系统调用的错误码 errno、进程退出的错误码retval,甚至 HTTP API 的状态码,都能看到这种方法的身影。

\paragraph{使用异常}

因为返回值不利于错误的传播,有诸多限制,Java 等很多语言使用异常来处理错误。

你可以把异常看成一种关注点分离(Separation of Concerns):错误的产生和错误的处理完全被分隔开,调用者不必关心错误,而被调者也不强求调用者关心错误。

程序中任何可能出错的地方,都可以抛出异常;而异常可以通过栈回溯(stack unwind)被一层层自动传递,直到遇到捕获异常的地方,如果回溯到 main 函数还无人捕获,程序就会崩溃。


\paragraph{使用类型系统}



\subsubsection{选择合理的错误处理方式}

Rust处理错误时,根据不同的错误采取不同的处理方式,选择最合理的方式来处理。Rust 将错误组合成两个主要类别:可恢复错误(recoverable)和 不可恢复错误(unrecoverable)。目前的原则是针对不可恢复的错误,以终止程序为主。可恢复的错误,打印日志为主。没有合适的错误处理原则,会让程序意外结束或者过度的嵌套的模式匹配,影响代码美观。例如获取Redis连接时,这样处理:

\begin{lstlisting}
let config_redis_string: String = env::var("REDIS_URL").expect("redis url missing");
\end{lstlisting}

这里获取Redis连接串失败时,直接退出应用,因为Redis连接串是需要强制存在的配置,无法获取到Redis连接串,后续的逻辑也无法完成。直接调用expect传入错误信息,结束运行。expect函数和unwrap函数功能点很接近,unwrap会打印出标准库内置的错误信息,而expect函数允许用户定义一个字符串,在程序结束的时候显示。


\end{document}