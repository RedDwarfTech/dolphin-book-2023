\documentclass[../../../dolphin-book-2023.tex]{subfiles}

\begin{document}

\subsection{Rust生命周期}

在Rust中每个引用都有自己的生命周期,生命周期保证了引用保持有效的作用域,在大多数情况下这个生命周期是隐式的,是可以被推断出来的,当引用的生命周期可能以不同的方式互相关联时,那就要手动标注生命周期了。

\subsubsection{悬垂引用}

生命周期最主要的作用就是避免悬垂引用,进而避免程序引用到非预期的数据。例子:

\begin{lstlisting}[language=Rust]
fn main() {
    let s = String::from("hello").as_str();
    println!("slice: {}", s); 
}
\end{lstlisting}

上面的代码在编译时会提示错误:temporary value dropped while borrowed,consider using a `let` binding to create a longer lived value。是因为as\_str只是持有String的引用,String在代码运行结束后会自动drop掉,导致s变成了悬垂引用。


\end{document}