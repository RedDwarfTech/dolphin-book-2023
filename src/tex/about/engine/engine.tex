\documentclass[../../../dolphin-book-2023.tex]{subfiles}

\begin{document}

\subsection{\LaTeX{}}

\LaTeX{}是一种基于\TeX{}的排版引擎。那么\TeX{}又是什么呢?\TeX{}是一个由美国计算机教授高德纳(Donald Ervin Knuth)编写的排版软件。TeX的MIME类型为application/x-tex,是一款自由软件。它在学术界特别是数学、物理学和计算机科学界十分流行。

20 世纪 60 年代,著名计算机科学家和数学家,斯坦福大学Donald Knuth教授在忙于撰写那部叫做《计算机程序设计艺术》 的书。这部书计划一共写七卷,Knuth 在写第四卷时,出版社拿来了第二卷的第二版书样给他过目,结果令他大失所望。因为刚刚出现的计算机排版的书籍质量实在无法令Donald Knuth教授满意。Donald Knuth教授的第一版书籍是按照传统的排版方式手工排版,由专业的排版人员实现,手工排版经过几个世纪的发展,能够排版出精美的书籍。但是19世纪60年代计算机也才出现没多久,C语言1972年才出现,从排版书籍的质量可以了解到,当时排版系统的能力暂时还不能够满足Donald Knuth教授的要求。于是Donald Knuth教授就计划开发一个符合自己要求的高质量的计算机排版系统。

这个排版系统的名字叫做\TeX{}。这个名称是由三个大写的希腊字母 ΤЄΧ 组成,在希腊语中是“科学”和“艺术”的意思。为了方便书写,一般在纯文本文档中将其写为 TeX, 念做“泰赫”(我个人习惯将其读作“泰克”)。

Knuth 于 1977 年开始构造 TeX 系统,并为该系统设计了一个字体生成软件——METAFONT。1982 年 TeX 系统发布,之后又有几次版本升级。Knuth 用圆周率 π 的近似值作为 TeX 系统的版本号,并采用自然底数 e 的近似值作为 METAFONT 版本号。系统每升级一次,其版号就增加一位数字,从而不断地趋近于 π 和 e,这种别出心裁的版本号表示方式一方面可以展示 TeX 与科技文献排版的密切关系,同时也表达了 Knuth 对 TeX 系统与 METAFONT 系统不断追求完美的愿望 。1990年 TeX 第3.1版发布时,Knuth 发出宣言:

\begin{itemize}
    \item 不再对 TeX 进行任何功能上的扩张。
    \item 如果出现明显问题,则修正后的版本号依次为3.14,3.141,3.1415.....在自己离开这个世界的时候,将最后的 TeX 版本序号改为 π。 此后,即使再发现错误,也都将成为 TeX 的特征而保留。如果有人非要修改的话,就不要再叫 TeX 了,请另外起名。
    \item 关于 TeX 的一切,已经全部做了书面说明,可以自由利用来设计其他的软件。
\end{itemize}

TeX 系统的内核相当稳定,几乎没有 bug,1995 年以后版本号一直停止在3.14159,直到 2002 年 12 月才又进行了一次升级。到目前为止,TeX 系统的版本序号是3.141592,而 METAFONT 版本序号则为 2.71828。

与时下家喻户晓的微软 Word 那种所见即所得的文档排版方式相比,TeX 显得多么的不合时宜,然而 Knuth 说:『我从来也不期盼 TeX 会成为一个万能的排版工具,让大家使用它可以来做一些「快速而脏」的东西;我只是将其视为一种只要你足够用心就能得到最好结果的东西。』

Knuth 之所以很自信的宣布不再对 TeX 进行任何功能上的扩张,并非自视甚高,而是有一定客观原因的。其中最主要的原因就是 TeX 提供了宏扩展机制,开发者或者用户均可以对 TeX 提供的 300 多条基本的控制序列(Control Sequence)进行组合,定义更为高级的控制序列,从而增强 TeX 的排版能力。

尽管 TeX 没有原生提供的功能可以通过宏扩展来实现,但是终究是有一些比较重要的问题是宏扩展方式难以解决的,比如对多国语言文字的支持、对 TrueType、OpenType 字体的支持、图形支持等问题。所以在 Knuth 的 TeX 之后,许多人努力地对 TeX 进行改进,或者干脆开发一个全新的“TeX”。开发者们为了尊重 Knuth 宣言的第二条,这些改进版的 TeX 或者重新开发的 TeX 均不再叫 TeX,它们都有新的名字,诸如 e-TeX、Omega、pdfTeX、XeTeX、LuaTeX 等。

自 1990 年以来,TeX 的改进项目层出不穷,但是大浪淘沙,许多项目都没有成功,有的已经死去,有的在苟延残喘。目前广为使用 TeX 改进版本主要有 pdfTeX、XeTeX 和 LuaTeX。应当注意的是 pdfTeX 项目开发者已于 2008 年 6 月宣布自 pdfTeX 1.50.0 版本之后只进行 bug 修正,不再提供新功能扩张。LuaTeX 项目则被看作是 pdfTeX 项目的延伸,并且添加了其它许多重要的新功能,并继续开发下去。

虽然 Knuth 的 TeX 系统出现了许多的改进版本和分化版本,但大都是良性的,并且变动的只是 TeX 引擎,它们大都兼容 Knuth 定义的 TeX 格式。这就好比只改进汽车 的发动机性能,并不改变或者只是略微改变汽车驾驶操作方式,这使得用户可以像往常一样去驾驶一辆性能更好的汽车。所以 TeX 用户们通常无需担心 TeX 引擎太多,导致自己的 TeX 文稿在其他人的计算机上无法编译。

\end{document}