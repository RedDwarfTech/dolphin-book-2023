\documentclass[../../../dolphin-book-2023.tex]{subfiles}


\begin{document}

\subsection{提示词工程最佳实践}

虽然我们可以根据自然语言随意的描述需求,但是时刻牢记并遵循如下原则或许有更大机会让AI生成更高质量的回答,提升自身与AI交互的体验,达到甚至超出我们的预期。本文档的绝大部分内容来源于OpenAI的提示词工程官方分享\footnote{\url{https://help.openai.com/en/articles/6654000-best-practices-for-prompt-engineering-with-openai-api}}。

\paragraph{使用最新的模型}为了获取更高质量的生成内容,推荐优先使用最新版本的模型。更新的模型可能改进了一些短板,比如降低了生成错误回答的概率,降低了胡说八道的概率。提升了模型的理解能力,提高了模型的响应速度等等。

\paragraph{使用\#\#\#或者"""符号来区分指令和文本}例如在需要总结一段文本时,不推荐的做法:

\begin{lstlisting}
请总结下面的文本。
文本内容......
\end{lstlisting}

推荐的做法:

\begin{lstlisting}
请总结下面的文本。
文本内容: """
balabala... 
"""
\end{lstlisting}

\paragraph{更具体}尽可能细致的描述您的需求,比如指出长度、格式、风格等等。

\paragraph{通过示例描述你的需求}

\paragraph{不要告知不能做什么,而是需要做什么}

\end{document}